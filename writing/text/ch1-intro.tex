\chapter{\label{ch:1-intro}Introduction} 

Information and communication technologies (ICTs) play a pivotal role in nearly all aspects of modern society. At the individual level, ICTs have transformed the way in which people communicate with each other, access information, conduct work, and manage finances (Roztocki et al., 2019). At the institutional level, ICTs have driven significant socio-economic advancements in various sectors such as agriculture, manufacturing, education, and government administration. These technologies have broadly improved service delivery, ensuring greater efficiency and accessibility in these critical areas (Roztocki \& Weistroffer, 2016). The extensive development and widespread adoption of ICTs in the UK are evident from the substantial increase in the number of secure internet servers per 1 million people, which surged from 1,315 in 2010 to 36,452 in 2020 (World Bank, 2024b). Additionally, there was a significant rise in the number of mobile cellular subscriptions per 100 people, from 74 in 2000 to 121 in 2022 (World Bank, 2024a). These statistics illustrate a profound shift towards a more connected and technologically integrated society.

Health and healthcare are no exception to the upward trend in ICTs utilisation. At the individual level, ICTs have significantly enhanced the ability to manage personal health. Information technologies such as search engines and specialised health information websites like the NHS Health A-Z (NHS, 2024) provide professional medical guidance on various diseases and symptoms. Additionally, wearable technologies like smartwatches enable real-time monitoring of key health indicators, such as heart rates and blood oxygen saturation. This continuous monitoring not only assists in improving health habits and the early detection of potential health issues but also facilitates the long-term treatment of chronic diseases (M. Wu \& Luo, 2019). At the institutional level, ICTs have fundamentally transformed the delivery of healthcare services. Traditional in-person services have shifted to digital platforms through electronic systems (eHealth) and mobile applications (mHealth) (Makri, 2019). These digital solutions offer numerous benefits, including the ability to scale information processing and streamline administrative processes. Technologies like electronic health records and online appointment and prescription applications enable healthcare providers to serve hundreds of patients simultaneously, which not only conserves resources but also eliminates the need for travel to in-person appointments when physical examinations or tests are not required (Alkureishi et al., 2021).

During the COVID-19 pandemic, the digitalisation of healthcare services was significantly reinforced and accelerated. Restrictions on social contact and mobility compelled many healthcare providers to suspend most non-emergency in-person services, prompting an increased reliance on digital platforms for health and social care access (Ramsetty\& Adams, 2020). According to Spanakis et al. (2021), registrations for the NHS app surged by 111\% from February to March 2020. The development and expansion of digital health technologies during COVID-19, including telehealth and telemedicine platforms, computerised clinical decision support systems, and contact tracing applications, have shown remarkable resilience in sustaining the healthcare system throughout this challenging period (Pandit et al., 2022). In the post-pandemic era, these digitalised healthcare solutions have not only become standard practice but are also set for continued enhancement (Getachew et al., 2023).

Despite the widespread adoption and increasing reliance on ICTs in modern health and healthcare systems, not everyone possesses the necessary skills to effectively engage with the Internet. Digital literacy, defined as the ability to use ICTs to find, evaluate, create, and communicate information (van Kessel, Wong, et al., 2022), exhibits significant variation across different segments of the population. This disparity, commonly referred to as the "digital divide" (Bernhardt, 2000; Hall et al., 2015; Makri, 2019), can lead to digital exclusion where individuals impacted by the divide may miss out on many benefits of digitalisation (Spanakis et al., 2021). The digital divide is most pronounced along generational lines, with older adults less likely to use the internet and other online technologies than younger people (Smith, 2014). According to the Labour Force Survey (LFS), while 92\% of UK adults were recent internet users in 2020, the rate for adults aged 16 to 44 was nearly universal at 99\%, but only 54\% for those aged 75 and over. The survey also estimates that over 3.1 million people aged 55 and older in the UK have never used the internet (Office for National Statistics, 2021). Furthermore, within the older population, there is still a noticeable age-based gradient; for instance, individuals aged 65-69 are more likely to engage with the Internet in their daily lives than those aged 80 and older (Anderson \& Perrin, 2017). In the context of health-related ICTs, findings from the 2018 National Health Interview Survey—a nationally representative survey of US households—show that only 38.9\% of older adults have used eHealth services (He et al., 2022).

The digital divide poses significant concerns about whether all individuals can equally benefit from the digitalisation of health and healthcare services. This issue is especially pronounced among older adults, who often lack sufficient digital literacy and simultaneously face greater healthcare needs (Eyrich et al., 2021). The rapid acceleration of digitalisation during the COVID-19 pandemic has heightened concerns that the digital divide will exacerbate existing health disparities, potentially leaving disadvantaged populations further behind. (Alkureishi et al., 2021). In response, the World Health Organization's "Global Strategy on Digital Health 2020-2025" emphasises the necessity of fostering an inclusive digital society. The strategy recommends two primary approaches to address these concerns: firstly, enhancing digital literacy among disadvantaged groups through improved internet access and targeted training; secondly, ensuring that digital health solutions are accessible to all, particularly the disadvantaged, to prevent digital health technologies from deepening existing inequalities (World Health Organization, 2021). 

Acknowledging the significance of this issue, this thesis seeks to explore the relationship between digital literacy and health outcomes among older adults. Specifically, by utilising data from the English Longitudinal Study of Ageing (ELSA), this study employs a quantitative methodology to address two key questions. First, it examines the causal effect of digital literacy on health outcomes among older adults. Second, it examines whether increased digitalisation will exacerbate the health inequalities between older adults with low and high levels of digital literacy.

The structure of this thesis is organised as follows: Section 2 reviews the existing literature on these two research questions and addresses their limitations respectively. Section 3 analyses the potential mechanisms and proposes the research hypotheses. Section 4 describes the analytical strategies of the two research questions in detail. Section 5 presents the results and discusses the findings in depth. Section 6 evaluates the contributions made by this thesis and also its limitations. Finally, Section 7 provides a summary of the study's main findings and discusses implications for further policy development.

