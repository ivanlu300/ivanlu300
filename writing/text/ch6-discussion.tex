\chapter{\label{ch:6-discussion}Discussion}

\section{Contribution of this thesis}
This thesis has made significant contributions to the existing literature, addressing the limitations outlined in Sections 2.3 and 2.4.

\subsection{Contribution of RQ1}
For the first research question, this thesis has advanced understanding in three key aspects:

\begin{enumerate}[wide=0pt, leftmargin=*, labelwidth=0pt, labelindent=\parindent, itemindent=0pt]
    \item Improved measurement of digital literacy \\
    As highlighted in Section 2.3.1, most existing secondary studies use frequency of Internet usage as a proxy for digital literacy when direct information on respondents' digital literacy level is unavailable. This approach, however, captures only a partial view of digital literacy. According to van DijK (2005, p. 21), digital literacy comprises both an instrumental dimension (i.e., hardware literacy) and an informational dimension (i.e., software literacy), yet frequency of Internet usage primarily reflects just the instrumental dimension, neglecting the informational aspect.

    This thesis advances the measurement of digital literacy by integrating both dimensions into a feature set that includes information on respondents' hardware literacy, software literacy, and frequency of Internet usage. Principal Component Analysis (PCA) is then employed to synthesise these features into a coherent index of digital literacy. This PCA-based approach effectively captures the broad spectrum of digital literacy, reflecting more accurately the variations in respondents' digital literacy levels. This methodology is validated by the loadings from the selected principal components, as evidenced in Table \ref{tab:pc1_loadings_rq1} and Table \ref{tab:pc1_loadings_rq2}. The loadings consistently indicate that a lower PC1 score represents a higher level of digital literacy, thereby providing a robust and nuanced measure that surpasses the conventional use of Internet usage frequency.

    \item Improved causal inference \\
    As highlighted in Section 2.3.2, the causal relationship between digital literacy and health is obscured by endogeneity issues, including omitted variable bias and simultaneity bias (Drentea et al., 2008; Y. A. Hong et al., 2017). However, the majority of existing studies only addresses the omitted variable bias through controlling for confounders. Thus, findings from these studies still present a biased estimate as simultaneity bias remains unaddressed.

    This thesis advanced the identification strategy by tackling both the simultaneity bias and omitted variable bias. Simultaneity bias is addressed by further removing respondents i) for whom health problems is one of the factors that prevent their Internet usage; or ii) who suffer from significant mobility limitations. These two types of respondents correspond to the two scenarios of simultaneity bias: poorer health leading to lower digital literacy and poorer health leading to higher digital literacy. Omitted variable bias is addressed through the implementation of the Inverse Probability of Treatment Weighting (IPTW) method. By weighting respondents based on the inverse probability of receiving the treatment, IPTW effectively removes the backdoor path of $\textnormal{digital literacy} \leftarrow \textnormal{confounders} \rightarrow \textnormal{health}$, enabling a more accurate estimation of the causal effects.

    \item Broader range of health outcomes \\
    As highlighted in Section 2.3.3, the range of health outcomes covered by existing studies is rather limited. Most of them have primarily concentrated on self-reported health and mental health outcomes. However, the aspect of physical health, which constitutes a crucial component of an individual's overall health, has received very limited attention. 

    This thesis expands the scope of investigated health outcomes by incorporating ten health metrics across three health dimensions: self-rated health, physical health (including three cardiovascular diseases: high blood pressure, high cholesterol, and diabetes, and three non-cardiovascular chronic diseases: asthma, arthritis, and cancer), and mental health (depression, anxiety disorder, and mood swings). This broadened approach offers a more complete evaluation of an individual's overall health, reflecting a wider spectrum of potential impacts of digital literacy.
\end{enumerate}

\subsection{Contribution of RQ2}
For the second research question, this thesis has advanced understanding in two key aspects:

\begin{enumerate}[wide=0pt, leftmargin=*, labelwidth=0pt, labelindent=\parindent, itemindent=0pt]
    \item Providing quantitative evidence \\
    As highlighted in Section 2.4.1, the body of existing research on whether increased digitalisation has exacerbated the health disparities between older adults with low and high digital literacy is notably scant. The only study by Alkureishi et al. (2021) provided only qualitative insights based on interviews with a modest sample of 54 participants, which limits the generalizability and depth of the findings.

    This thesis significantly enhances the literature by introducing robust quantitative evidence. Exploiting the exogenous increase in digitalisation induced by the COVID-19 pandemic, this study leverages data from Waves 9 and 10 of the English Longitudinal Study of Ageing (ELSA). The timing of Wave 9, conducted pre-pandemic, and Wave 10, conducted during the pandemic, creates an ideal framework to examine the dynamics of health disparities as influenced by the sudden shift towards greater digitalisation. This unique dataset, combined with a rigorously designed identification strategy, offers substantial quantitative evidence to explore the evolution of health disparities, making a compelling case for the impact of increased digital engagement on older adults.

    \item Improved causal inference \\
    As highlighted by Section 2.4.2, applying a simple Difference-in-Difference (DiD) analysis will not yield an accurate estimate of how increased digitalisation impacts the health disparities between older adults with low and high digital literacy. This stems from two significant challenges: the contamination of group assignment and the violation of the parallel trends assumption (Chetty et al., 2018; Hall et al., 2015).

    This thesis advances the identification strategy by tackling both biases. To address the possible contamination of group assignment, separate PCA analyses were conducted for Wave 9 and Wave 10. By doing so, the study ensures only respondents whose digital literacy status remains constant across the two waves are included in the analysis, thereby maintaining consistent group assignments. The potential violation of the parallel trends assumption is addressed through the implementation of the Matching DiD method. This approach involves matching each respondent in the low digital literacy group with a closely ``similar” counterpart in the high digital literacy group. DiD analysis is then individually applied to each matched pair, and the collective results are aggregated to derive the final impact estimate. By only comparing ``similar” respondents, this study increases the likelihood that the parallel trends assumption holds and enhances the reliability of the findings. 
\end{enumerate}

\section{Limitations}
Results of this thesis should be critically assessed in light of the following limitations.

\subsection{Reification of principal components}
As described in Section 4.3.1.2, this study utilises the principal component(s) derived from PCA as the digital literacy index. It is important to recognise that PCA is an unsupervised machine learning technique that summarises the variance within a dataset. While powerful, it is crucial to acknowledge that the identified principal components themselves lack intrinsic meaning — they are merely statistical constructs derived to capture significant patterns in the data (James et al., 2023, p. 509).

In this study, the first principal component (PC1) has been interpreted as representing respondents' varying levels of digital literacy. Considerable efforts have been made to substantiate this interpretation. Firstly, the feature set chosen for PCA was deliberately selected to encapsulate the core aspects of digital literacy, ensuring that the variance captured is pertinent to this concept. Secondly, to mitigate potential misinterpretations, this study pre-emptively eliminated the influence of internet access disparities — a potential alternative interpretation of PC1 — by excluding respondents without Internet access at home before conducting the PCA. Thirdly, the interpretation is further supported by an examination of its loadings, which uniformly indicate a correspondence with digital literacy.

However, it is important to emphasise that this act of assigning a specific meaning to PC1 — an abstract mathematical output — is an example of reification (Jolliffe, 2002). Although this study has demonstrated that there are strong reasons to believe that PC1 validly represents digital literacy, there remains a risk that this interpretation over reifies this mathematical construct. Thus, caution should be exercised, and the findings interpreted with an understanding that they are contingent upon the assumption that PC1 adequately represents digital literacy.

\subsection{Lack of robustness check for RQ2}
Another limitation of this study is the inability to conduct robustness checks for the second research question, primarily due to data constraints. A typical robustness approach in DiD analyses involves conducting a placebo test. This test simulates the timing of the intervention, in this case, increased digitalisation, to check for spurious effects (Slusky, 2017). Conventionally, one might fake the treatment to have occurred one wave earlier (i.e. between Wave 8 and Wave 9) instead of the real time (i.e. between Wave 9 and 10), and then apply the DiD methodology to these waves. Under ideal conditions, no significant effects should be observed since the treatment is fictitious. Observing significant results in such a scenario would call into question the validity of the original findings.

However, implementing this placebo test is impractical for this study. As noted in Section 4.2, this study relies on information on respondents' hardware and software literacy to perform PCA and obtain the digital literacy index. Unfortunately, this information only becomes available starting from Wave 9. Without equivalent data from Wave 8, it is impossible to apply the same group segmentation criteria used in the DiD analysis, hence precluding a placebo test for earlier waves. This restriction underscores the need for caution in interpreting the robustness of the findings from the second research question.

\subsection{Lack of explanation for inconsistency between health outcomes}
This study encompasses a broad range of health outcomes, selecting in total ten metrics across three health dimensions to ensure comprehensiveness. However, findings are not consistent among these outcomes. Specifically, for the first research question, statistically significant effect is found in 6 out of 10 health outcomes, and for the second research question, statistically significant effect in found in 3 out of 10 health outcomes.

This inconsistency across different health outcomes raises important questions. For instance, for the first research question, digital literacy significantly decreases the likelihood of being diagnosed with high cholesterol, but shows no effect on arthritis. However, the mechanisms proposed in Sections 3.1 and 3.2 fail to offer a convincing explanation for this difference. They only provide a general framework for understanding how digital contribute to health outcomes. The concepts of cultural capital and social engagement apply broadly to all three health dimensions – self-rated health, physical health and mental health — but they are not specific to any individual health condition. This lack of specificity limits their utility in explaining why certain health outcomes are affected by digital literacy while others are not.

Understanding why digital literacy affects various health outcomes differently is crucial for public health. Identifying the specific mechanisms by which digital literacy influences each disease can guide the development of targeted interventions to optimise the benefits of digital literacy enhancements. This tailored approach enables policymakers to formulate strategies that address the precise needs of specific populations, ensuring that improvements in digital literacy translate into substantial health benefits (Harris \& McDade, 2018). Such detailed investigations enrich both academic understanding and practical public health initiatives.
