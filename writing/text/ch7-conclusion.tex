\chapter{\label{ch:7-conclusion}Conclusion}

\section{Summary of main findings}
This thesis embarked on an extensive exploration into the implications of digital literacy on health outcomes among older adults, framed within the rapidly advancing context of digitalisation in healthcare and everyday life. Specifically, this thesis seeks to address two key research questions. First, what is the causal effect of digital literacy on health outcomes among older adults? Second, whether increased digitalisation has exacerbated the health inequalities between older adults with low and high digital literacy?

A significant challenge in this research was the accurate measurement of digital literacy among older adults, an area where previous literature has often been inadequate. To overcome this, the study employed a sophisticated analytical strategy using Principal Component Analysis (PCA), an unsupervised machine learning algorithm. This technique was instrumental in extracting principal components that encapsulate the respondents' digital literacy from a multifaceted set of features that are intricately linked to the core concepts of digital literacy.

To address the first research question, this thesis utilised cross-sectional data from Wave 10 of the English Longitudinal Study of Ageing (ELSA). In response to critiques of existing studies, which often neglect the complexities of simultaneity bias and omitted variable bias, this study implemented several methodological corrections. Notably, the analysis strategically excluded certain respondents to alleviate simultaneity bias and utilized Inverse Probability of Treatment Weighting (IPTW) to counteract omitted variable biases. Furthermore, a robustness check (sensitivity analysis) was conducted to determine the robustness of the findings against potential violations of the strong ignorability assumption inherent in IPTW. The main findings from the analysis of the first research question indicated that high digital literacy significantly benefits health outcomes in six out of the ten health metrics evaluated. These beneficial effects of digital literacy on health were found to be robust, even under stringent tests for potential violations of underlying assumptions of the analytical models used.

For the second research question, this thesis leveraged panel data from Waves 9 and 10 of ELSA, which provided a unique opportunity to examine the impact of an exogenously induced escalation in digitalisation due to the COVID-19 pandemic. Addressing potential methodological shortcomings noted in earlier studies, such as contamination of group assignments and violations of the parallel trends assumption, this study employed rigorous corrections. Specifically, the analysis removed individuals whose group status changed between waves to maintain consistency of group assignment and applied a Matching Difference-in-Difference (Matching DiD) approach to reinforce the likelihood of the parallel trends assumption holding true. The analysis of the second research question revealed that increased digitalisation has significantly widened health disparities between older adults with low and high digital literacy in three out of the ten health outcomes examined. 

\section{Policy implications}
The findings of this thesis elucidate the significant impact of digital literacy on health outcomes among older adults and underscore the expanding disparities caused by rapid digitalisation in healthcare. These insights provide a solid basis for shaping targeted policy interventions aimed at narrowing these disparities and enhancing the overall well-being of this demographic.

The research indicates that higher digital literacy has a clear and significant beneficial effect on health outcomes. This robust effect emphasises the necessity for policies that enhance digital skills among older adults, integrating digital literacy training into regular health management and preventive care. Addressing this requires a collaborative approach involving health professionals, policy makers, and technology experts to create educational programmes that cater specifically to the needs of older adults (Eyrich et al., 2021).

Moreover, the research clearly shows that increased digitalisation has deepened the health disparities between older adults with varying levels of digital literacy. It is imperative that digital health solutions are made accessible to everyone, especially the most disadvantaged, to prevent these technologies from widening existing inequalities. This effort should involve both public and private sectors in developing digital technologies suited for older adults, which could include designing user-friendly digital platforms with simplified interfaces and improved readability. Additionally, incentives should be in place for developers to create digital tools that are straightforward for this age group to navigate, addressing common usability issues.

Although the central focus of this thesis is on digital literacy, it is crucial to acknowledge that a significant segment of the population still lacks basic access to the Internet. This digital divide poses a fundamental barrier to achieving the benefits of digital literacy, particularly for older adults who may be isolated due to socioeconomic or geographic constraints. To bridge this gap, addressing economic barriers is imperative. Subsidies or targeted government programs designed to make digital devices and Internet services more affordable can play a pivotal role in ensuring equitable access to digital resources. These initiatives should not only focus on reducing the cost of technology but also on providing comprehensive support services that help older adults navigate these digital tools effectively (Makri, 2019).

Ongoing research and evaluation are vital to maintain the efficacy of these policies as digital technologies continue to evolve. As new technologies emerge and additional data become available on their effectiveness, it is crucial that policies are adaptable and can integrate these advancements into health management strategies for older adults effectively.

In conclusion, the policies recommended in this thesis advocate for a comprehensive, inclusive approach to digital literacy, ensuring that older adults are not merely passive recipients of digital education but active participants in a digitally evolving healthcare landscape. By creating an environment where digital skills are consistently nurtured and where digital tools are readily accessible, we can significantly reduce health disparities and improve the quality of life for older adults in our digital age. This strategy not only enhances individual health outcomes but also contributes to a more equitable and inclusive digital healthcare system.
