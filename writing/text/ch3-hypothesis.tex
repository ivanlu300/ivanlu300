\chapter{\label{ch:3-hypothesis}Mechanisms and hypotheses}

As highlighted by analytical sociology, a profound understanding of the social world requires more than just identifying statistical associations; it demands an exploration of the underlying mechanisms that drive these relationships. \textcite[p. 5]{hedstrom_what_2011} emphasise the importance of making empirical regularities ``not only visible but transparent" by uncovering the mechanisms most likely leading to meaningful explanations. \textcite[p. 112]{goldthorpe_sociology_2016} further elaborates that a causal explanation necessitates detailing the continuous time-space mechanisms through which a supposed cause actually produces its effect. 

In light of these perspectives, this section will first examine the potential mechanisms underpinning the two research questions, drawing on relevant theories to provide a deeper contextual understanding. Following this theoretical exploration, the section will present corresponding hypotheses for each research question, thereby structuring a clear and empirically testable framework for further analytical investigation.

\section{Mechanisms - RQ1}
There are two potential mechanisms through which digital literacy can impact the health outcomes of older adults. 

\subsection{Digital literacy as cultural capital}
The first mechanism is informed by Bourdieu's capital theory, which posits that individuals possess three forms of capital — economic capital, social capital, and cultural capital \parencite{bourdieu_forms_1986}. Economic capital refers to an individual's material wealth and financial assets. Social capital pertains to the network of institutionalised relationships of mutual acquaintance that an individual can draw upon. Cultural capital, meanwhile, encompasses the non-material assets possessed by individuals, serving as a symbolic indicator of socio-economic status. Cultural capital has three different forms \parencite{xu_individual_2020}:

\begin{enumerate}
    \item The embodied state, which includes dispositions and habits acquired throughout life, such as dietary preferences or exercise routines.
    \item The institutionalised state, which relates to formal certificates recognised by the society, such as academic degrees.
    \item The objectified state, which exists in the form of cultural goods such as the possession of books and artworks.
\end{enumerate}

According to Bourdieu, capital shapes people's beliefs and behaviours and ultimately determine their life outcomes and positions within the social hierarchy \parencite{xu_individual_2020}. Empirical evidence robustly supports this theory, demonstrating clear correlations between cultural capital and health: i) individuals with more cultural capital typically engage in behaviours that are more beneficial to health, such as increased physical activity and reduced smoking and alcohol consumption \parencite{shim_cultural_2010}; ii) individuals with more cultural capital tend to experience better health outcomes, including lower rates of mortality and morbidity, as well as improved health-related quality of life \parencite{abel_cultural_2008}.

In this context, digital literacy can be regarded as a form of cultural capital, specifically falling within the embodied state of cultural capital. It fundamentally shapes how individuals access and navigate the digital realm and significantly enhances their ability to engage with digital health resources and technologies, thereby contributing to improved health outcomes \parencite{adkins_health_2009}. This health-promoting effect of digital literacy as a form of cultural capital manifests in three key ways: First, digital literacy enables individuals to access valuable health information on the Internet \parencite{sorensen_health_2015}. As an increasing amount of information becomes available online, when facing mild health issues such as coughs or rashes, individuals with digital literacy skills are better positioned to consult reliable online health resources (e.g. NHS Health A-Z). This capability allows them to gain a deeper understanding of their symptoms and evaluate potential treatments, enabling them to make informed decisions about over-the-counter medications or when to seek professional medical advice \parencite{aboueid_use_2019}.

Secondly, digital literacy enables individuals to utilise health-promoting technologies. The rapid advancement of health-related ICTs, including wearable devices and the Internet of Things (IoT), greatly expands individuals' capabilities to monitor and manage their own health effectively \parencite{schulz_advancing_2015}. For healthy individuals, using smart watches for self-monitoring of personal activities offers substantial health benefits. For example, sleep tracker applications can significantly enhance one's understanding of their sleep patterns, enabling them to make informed adjustments to improve sleep quality, which has been proven to enhance overall health outcomes by promoting better mental and physical well-being \parencite{reeder_health_2016}. For those managing long-term chronic conditions, health-related ICTs provide essential capabilities for monitoring vital health indicators, which can be crucial for detecting significant changes that may indicate the need for urgent medical intervention. For instance, a person with heart disease might use an IoT device (e.g., body censor) to monitor heart rate and rhythm, enabling early detection of potential cardiac events. This timely information can be vital, allowing for swift medical responses that might prevent conditions from deteriorating \parencite{domingo_overview_2012}.

Third, digital literacy enables individuals to effectively interact with an increasingly digitalised healthcare system. Individuals proficient in digital skills can conveniently book appointments with their General Practitioners (GPs), order prescriptions, and access personal medical records online, thereby streamlining their healthcare management and more rapidly addressing their health concerns \parencite{makri_bridging_2019}. This capability is particularly crucial during times when traditional, face-to-face service delivery is disrupted, such as during the COVID-19 pandemic. During this period, the importance of digital literacy as a form of cultural capital was particularly evident, as those with these skills could seamlessly transition to virtual healthcare services \parencite{eyrich_bridging_2021}. This transition not only ensured continued access to medical care but also facilitated more efficient and timely healthcare interactions. Through telemedicine and video-enabled consultations, digitally literate patients could maintain regular contact with healthcare providers, greatly enhancing the convenience, timeliness, and effectiveness of the healthcare services they received \parencite{crawford_digital_2020}.

In summary, digital literacy, as a form of cultural capital, not only enhances individuals' access to health information and technologies but also improves their ability to interact with an increasingly digitalised healthcare system, collectively contributing to improved health outcomes.

\subsection{Digital literacy empowers social engagement}
The second mechanism is informed by the notion of ``active ageing”. Active ageing theory suggests that there is a beneficial relationship between the level of social engagement in older adults and their health outcomes \parencite{walker_active_2012}. Studies have consistently confirmed that active social engagement correlates with improved health outcomes in several aspects, including cardiovascular diseases \parencite{ramsay_social_2008}, disabilities \parencite{mendesdeleon_social_2003}, and depressive symptoms \parencite{glass_social_2006}.

In today's world, social engagement is no longer confined to the traditional offline environments; there is a burgeoning online world that is increasingly capturing the engagement of a wider population. Many social activities traditionally conducted offline have now expanded or even transitioned entirely to the online sphere \parencite{lieberman_two_2020}. For instance, maintaining friendships no longer depends on face-to-face meetings; rather, interactions through online chat applications and engagement through posting content and commenting on social media platforms have become key methods of connecting with the community \parencite{shklovski_friendship_2015}. This shift became particularly pronounced during the COVID-19 pandemic, when traditional forms of social engagement were drastically curtailed. With physical distancing measures in place and large gatherings banned, the online world not only remained accessible but became a crucial lifeline, serving as the primary — or in some cases, the only — means for people to maintain social connections and community ties \parencite{ammar_covid19_2020}. 

Given the significant role the online world now plays in people's social engagement, digital literacy has become an essential prerequisite for a full participation in social interactions. Digital literacy enhances social engagement and through this channel, it also benefits older people's health outcomes in two key ways. First, it provides a crucial avenue for social support. Engaging in the online world offers an additional layer of social support beyond what is available offline. Friends and networks online can provide connectedness, feedback, and consulting, which collectively improve well-being by alleviating loneliness and enhancing self-confidence and self-esteem \parencite{czaja_improving_2018}. Moreover, the unique quality of the Internet - anonymity - allows individuals to interact without the biases and barriers often encountered in face-to-face communications, enabling more open and honest exchanges. Such interactions can lead to stronger, more meaningful connections that are essential for emotional support and overall well-being in later life \parencite{wangberg_relations_2008}.

Second, digital literacy enhances the potential for social influence through online social engagement. The concept of interpersonal influence within close social networks is well-documented in social science research \parencite{zagenczyk_social_2010}. Individuals often align their behaviours with those of a reference group — typically their peers — by comparing and adjusting their actions based on observed norms and guidance. Behaviours tend to be reinforced when they align with those of the reference group and modified when they deviate \parencite{marsden_network_1993}. In online social communities, this dynamic can be particularly influential if the environment promotes positive health behaviours. For instance, when an online community consistently promotes and endorses health-enhancing practices, such as regular exercise routines and timely healthcare utilisation, its members are more likely to embrace these positive behaviours. As a result, such social engagement can lead to improved health outcomes among the community members \parencite{poirier_social_2012}.

It should be noted that while this thesis identifies two potential mechanisms linking digital literacy with older adults' health outcomes, it does not imply that these mechanisms function entirely independently of one another. Rather, they are likely interconnected and may act synergistically to enhance the overall impact of digital literacy on health outcomes. Digital literacy as a form of cultural capital does more than just facilitate access to health information and resources; it also facilitates social engagement within the digital realm, further contributing to improved health outcomes through enhanced social support and influence. Essentially, digital literacy acts as a pivotal tool that intersects with both the cultural capital and social engagement paradigms, offering a comprehensive framework for understanding its significant influence on the health trajectories of older adults.

\section{Mechanisms - RQ2}
The mechanism by which increased digitalisation may have exacerbated health disparities between older adults with low and high levels of digital literacy is principally grounded in Bourdieu's capital theory, as explored in Section 3.1.1. Literature consistently shows that as the significance of cultural capital escalates, so does the disparity between those who possess such capital and those who do not \parencite{bennett_introduction_2006}. For instance, \textcite{reay_education_2004} observed that the expansion of policy initiatives emphasising the role of parents in primary education has made the impacts of parental background on educational outcomes more evident and in turn exacerbated the existing educational inequalities along this line. 

As discussed in Section 3.1.1, digital literacy is a crucial form of cultural capital. With the progression of digitalisation, digital literacy assumes a more pivotal role due to its broader application and deeper integration into daily life. For example, prior to the COVID-19 pandemic, consultations with a General Practitioner (GP) could be conducted either in-person or online. However, during the pandemic, consultations shifted predominantly or even solely online \parencite{crawford_digital_2020}. This shift has significantly highlighted the growing importance of digital literacy. In scenarios where digital platforms become the main gateway for essential services, the digital divide will become increasingly evident and problematic. Consequently, the health disparities between individuals with high and low digital literacy are likely to widen further.

\section{Hypotheses}
Building on the analysis of the potential mechanisms underpinning the two research questions, this thesis proposes the following hypotheses:

\begin{itemize}
    \item Hypothesis 1 (for RQ1): Digital literacy has a beneficial effect on older adults' health, meaning that higher levels of digital literacy will lead to improved health outcomes.
    \item Hypothesis 2 (for RQ2): Increased digitalisation will exacerbate the health disparities between older adults with low and high digital literacy.
\end{itemize}