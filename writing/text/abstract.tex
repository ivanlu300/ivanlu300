\begin{itemize}
    \item \textbf{Background}: As digital technologies increasingly permeate all aspects of society, particularly healthcare, understanding the role of digital literacy in shaping health outcomes among older adults becomes crucial. This thesis investigates the impact of digital literacy on health outcomes and assesses whether heightened digitalisation has exacerbated health disparities between older adults with varying levels of digital literacy.
    \item \textbf{Methodology}: Utilising data from the English Longitudinal Study of Ageing, this study employs quantitative analysis techniques to explore the relationship between digital literacy levels and various health outcomes among older adults. It specifically examines the causal effect of digital literacy on health through Inverse Probability of Treatment Weighting and investigates whether increased digitalisation exacerbates health inequalities using a Matching Difference-in-Difference approach.
    \item \textbf{Results}: The findings suggest that higher levels of digital literacy lead to better health outcomes, indicating that proficiency in digital tools is integral to managing health effectively in a digital age. Additionally, the study reveals that increased digitalisation tends to widen health disparities among those with different levels of digital literacy, highlighting a growing divide in health equity related to digital skills.
    \item \textbf{Conclusion}: This thesis underscores digital literacy as a significant determinant of health among older adults, providing evidence for policymakers to foster digital education and skill development to bridge health disparities. By enhancing digital literacy, health equity can be improved, making the benefits of digital healthcare more universally accessible.
\end{itemize}