\chapter{\label{ch:2-litreview}Literature review}

\section{Existing studies - RQ1}
The first research question addresses the causal effect of digital literacy on health. A number of researchers have explored this relationship, particularly over the past decade as concerns about the digital divide have grown considerably (Arias López et al., 2023). This section will provide an overview of key studies, focusing firstly on the types of health outcomes examined and secondly on the identification strategies employed.

\subsection{Types of health outcomes}
It is widely acknowledged that health encompasses two dimensions: physical health and mental health (Department of Health, 2014; World Health Organization, 2024). In terms of health measurement, a further distinction is made between self-rated and objective measures (Baker et al., 2004; S. Wu et al., 2013). The literature consistently shows that poorer digital literacy is significantly associated with worse outcomes across all three of these aspects.

Firstly, regarding self-rated health, Gracia and Herrero (2009) analysed a survey among older people in Spain, and self-rated health is measured on a five-point Likert scale in the survey. They identified a significant association between Internet use and a decreased likelihood of reporting poor health (OR = 0.41, p = 0.02), indicating that Internet users generally report better self-rated health than non-users. Similarly, Falk Erhag et al. (2019) analysed cross-sectional data from the Gothenburg H70 Birth Cohort Study in Sweden, again using a five-point Likert scale to measure self-rated health. They found a significant correlation between more frequent Internet use and better self-rated health ($\beta = 0.101, p < 0.001$). A similar trend has been observed in Asia; Hong et al. (2017) drew on cross-sectional data from the China Health and Retirement Longitudinal Study (CHARLS). They employed a binary version of self-rated health and discovered that Internet users are more likely to report good health compared to non-users ($OR = 1.727, p < 0.001$).

Secondly, regrading physical health, the current literature is notably scarce. From what has been surveyed (Arias López et al., 2023), only two studies have specifically examined the relationship between digital literacy and physical health, focusing on diabetes and lung disease respectively. Guo et al. (2021) conducted a cross-sectional survey among 249 diabetes patients across three hospitals in Taiwan, measuring their digital health literacy through a self-developed questionnaire. Their findings indicate that higher digital health literacy has a direct, albeit non-significant, effect on reducing glycated haemoglobin (HbA1c) levels — a lower HbA1c level signifies better diabetes control. Stellefson et al. (2019) carried out a survey among 174 chronic obstructive pulmonary disease (COPD) patients in the US, using the 8-item eHealth Literacy Scale (eHEALS) (Norman \& Skinner, 2006) to assess their digital health literacy. They discovered that patients with higher digital health literacy are significantly less likely to experience symptoms such as chest congestion and tightness, and have significantly better sleep quality.

Thirdly, regarding mental health, Lee et al. (2018) analysed data from the US National Health and Aging Trends Study, where depression was measured using the Patient Health Questionnaire 2 (PHQ-2) (Kroenke et al., 2003). They found that greater use of ICTs is significantly associated with a reduced risk of severe depressive symptoms ($OR = 0.28, p < 0.05$). Similarly, Yang et al. (2021) conducted a web-based cross-sectional survey in China during COVID-19, assessing respondents' digital health literacy with a self-developed questionnaire. Their findings indicate that higher digital health literacy is inversely correlated with various mental health outcomes, including depression ($r = -0.331$), insomnia ($r = -0.366$) and post-traumatic stress disorder (PTSD) ($r = -0.320$).

In summary, the studies reviewed consistently demonstrate that lower digital literacy is associated with poorer health outcomes. This uniformity suggests a broad consensus in the literature that digital literacy may be a crucial determinant of health among older adults (Arias López et al., 2023; van Kessel, Wong, et al., 2022). However, whether these findings reflect a valid causal effect requires further investigation. A more detailed discussion on this matter will be provided in Section 2.3.2.

\subsection{Identification strategies}
When investigating the causal effect of digital literacy on health outcomes, most existing studies employ the "selection on observables" approach, typically within a regression framework, to derive their causal estimates. This method involves controlling for potential confounders that are correlated with both digital literacy and health outcomes. Common control variables frequently used in these analyses include age, gender, marital status, race, educational attainment, employment status and family income (Hall et al., 2015; He et al., 2022; Mitchell et al., 2019).

The predominant methodology among these studies is the use of straightforward OLS or logistic regression models that regress health outcomes on digital literacy along with individual-level control variables. The coefficient attributed to digital literacy in these models is often interpreted as the estimated causal effect on health. For instance, when analysing the effect of digital literacy on self-rated health, Gracia and Herrero (2009) utilised a logistic regression model when analysing the effect of Internet usage on self-rated health, incorporating controls for age, gender, marital status, and area of residence. 

Some research extends beyond basic models by incorporating more sophisticated methods that control for confounders at multiple levels. For example, Hong et al. (2017) used a multi-level logistic regression model when assessing the effect of Internet usage on self-rated health, including both individual-level variables and community-level factors such as neighbourhood amenities, healthcare facilities, and community organisations. 

However, it is important to note that relying solely on "selection on observables" may not yield a valid causal estimate due to the potential for simultaneity bias (i.e., reverse causality), which can significantly skew the relationship between digital literacy and health (Czaja et al., 2013; Drentea et al., 2008; Greer et al., 2019). A more comprehensive discussion of these limitations and their implications will be explored in Section 2.3.2.


\section{Existing studies - RQ2}
The second research question addresses whether increased digitalisation will exacerbate the health inequalities between older adults with low and high levels of digital literacy. Although many researchers advocate that digital health ecosystems should prioritise health equity to prevent worsening existing disparities (Eyrich et al., 2021; van Kessel, Hrzic, et al., 2022), empirical evidence on this issue remains scarce in the literature. While most studies have examined health disparities at a single point in time (like the ones analysed in Section 2.1), there is a lack of research exploring how these inequalities evolve in conjunction with the growing digitalisation of healthcare. To the best of my knowledge, only one study has delved into this dynamic, employing qualitative interviews to explore how increased digitalisation impacts health inequalities among older adults.

Alkureishi et al. (2021) conducted 54 telephone interviews with adult patients at the University of Chicago Medical Center between December 2020 and March 2021, focusing on their experiences with the digital divide and the expansion of ICT-dependent healthcare during the COVID-19 pandemic. Two participants noted that the increasing reliance on ICTs lengthened their wait times for medical visits. They observed that individuals who could engage in virtual visits accessed care more quickly, including faster responses to patient portal messages and more readily available virtual appointments. Additionally, other participants reported that the escalating digitalisation of healthcare services negatively impacted their personal health outcomes due to limited access and skills needed to coordinate their healthcare effectively.

\section{Limitations of existing studies - RQ1}
The findings from the existing literature on the causal effect of digital literacy on health should be evaluated considering the following limitations.

\subsection{Measurement of digital literacy}
As van DijK (2005, p. 21) proposes, digital literacy comprises two key dimensions: instrumental and informational. The instrumental dimension pertains to hardware literacy, focusing on an individual's ability to operate devices such as mobile phones, tablets, and computers to connect to the Internet. The informational dimension, on the other hand, pertains to software literacy. It focuses on individuals' ability to effectively search, select, process, and utilise information from an abundance of sources to accomplish relevant tasks. Typical activities within this dimension include searching for specific information, engaging in social networking, managing finances, navigating directions, and using digital services (Chetty et al., 2018).

Measuring digital literacy is a complex challenge, as it encompasses a person's inherent abilities that are not directly quantifiable like educational attainment or as easily observable as gender. In most primary studies where researchers have the capacity to design and administer their own surveys, a standard approach is to use specially crafted and validated questionnaires that assess both the instrumental and informational dimensions of digital literacy. The 8-item eHealth Literacy Scale (eHEALS), developed by Norman and Skinner in 2006, is a widely used instrument that has been extensively validated and is considered highly accurate in measuring digital literacy (Arias López et al., 2023). An example of this approach can be seen in the study described above by Stellefson et al. (2019), where they conducted a primary survey among 174 COPD patients and incorporated the eHEALS scale to measure participants' digital literacy.

However, in secondary research where researchers rely on pre-existing datasets, measuring digital literacy becomes particularly challenging if the data itself lacks specifically designed questions for this purpose. Under these circumstances, researchers must devise innovative and credible methods to assess digital literacy effectively, ensuring that their approaches encompass both dimensions of the digital literacy (van Kessel, Wong, et al., 2022). This often requires a careful adaptation of available variables or the development of proxy measures that can accurately reflect the underlying variation in respondents' level of digital literacy.

Most of the existing secondary studies described in Section 2.1 have resorted to using frequency of Internet usage as a proxy measure for digital literacy. However, this approach falls short in capturing the full scope of digital literacy, as it predominantly reflects only the instrumental dimension — merely the ability to access and use the Internet. This measure fails to account for significant variations in the informational dimension, which encompasses the ability to perform a wide array of complex online tasks. For instance, imagine two respondents with the same frequency of Internet usage, one of them uses the Internet solely for chatting with friends whereas the other also uses the Internet for information searching, appointment booking and grocery shopping. It is clear that the latter has a much higher level of digital literacy than the former, but this difference cannot be sufficiently captured simply from their frequency of Internet usage. Consequently, the prevalent approach of measuring digital literacy in existing secondary research lacks sufficient accuracy and validity, thereby compromising the reliability of findings when applied to analyse the effect of digital literacy on health.

\subsection{Causal inference}
It is crucial to understand that the significant association between digital literacy and health outcomes observed in many studies does not necessarily imply causation. The causal relationship between digital literacy and health is obscured by endogeneity issues, including omitted variable bias (Y. A. Hong et al., 2017; S. S. M. Lam et al., 2020) and simultaneity bias (Drentea et al., 2008; L. Lam \& Lam, 2009). These biases mean that the observed bivariate associations might not accurately reflect the true causal impact of digital literacy on health. Consequently, without appropriate controls and methodological approaches to address these biases, conclusions drawn from these associations regarding causality may be misleading.

Omitted variable bias (OVB) occurs when a model fails to include one or more relevant variables that influence both the explanatory variable and the outcome variable (Angrist \& Pischke, 2009, p. 59). As illustrated in Figure \ref{fig:ovb}, if the uncontrolled confounder Z is positively or negatively correlated with both the explanatory variable X and the outcome variable Y, the effect of X on Y will be overestimated. Conversely, if Z is positively correlated with one variable and negatively correlated with the other, the effect of X on Y will be underestimated. 

\begin{figure}[h!]
    \centering
    \caption{Omitted variable bias}
    \label{fig:ovb}
    \begin{tikzpicture}[thick]
        \node (X) {X};
        \node (Z) [below right = of X] {Z};
        \node (Y) [above right = of Z] {Y};
          
        \draw[->] (Z) -- (X);
        \draw[->] (Z) -- (Y);
        \draw[->] (X) -- (Y);
      \end{tikzpicture}
\end{figure}

In the context of digital literacy and health, there are a number of widely recognised confounding variables in the literature. These include:

\begin{itemize}
    \item Age: Older age is associated with lower digital literacy and poorer health outcomes (Hall et al., 2015).
    \item Gender: Males on average have higher digital literacy and poorer health outcomes compared to females (He et al., 2022).
    \item Marital status: Married individuals tend to have higher digital literacy and better health outcomes (He et al., 2022).
    \item Ethnicity: The effect of ethnicity remains contested. Some suggest that racial disparities contribute to higher digital literacy and improved health outcomes among White individuals compared to non-White individuals (Mitchell et al., 2019). Conversely, the "healthy immigrant effect", which involves a selection bias where immigrants may be healthier and more digitally literate than the general population, supports the view that non-White individuals might have higher digital literacy and better health outcomes than their White counterparts (McDonald \& Kennedy, 2004).
    \item Educational attainment: Higher educational level is associated with higher digital literacy and better health outcomes (Hall et al., 2015).
    \item Employment status: Employed individuals tend to have higher digital literacy and better health outcomes (He et al., 2022). 
    \item Household Income: Higher household income is associated with both higher digital literacy and better health outcomes (He et al., 2022).
    \item Deprivation: More deprived individuals tend to have lower digital literacy and poorer health outcomes (He et al., 2022).
    \item Cognitive functioning: Better cognitive functioning is associated with both higher digital literacy and better health outcomes (J. W. Hong et al., 2023).
\end{itemize}

It is important to recognise that the list of confounders provided is not exhaustive. There may be additional observable and/or unobservable confounding variables. In theory, failing to control for even a single variable that influences both digital literacy and health outcomes could introduce omitted variable bias.

Simultaneity bias, also known as reverse causality, occurs when the relationship between variables is bidirectional; that is, not only does the explanatory variable affect the outcome variable, but the outcome variable also influences the explanatory variable (Freedman, 2005, p. 63). As illustrated in Figure \ref{fig:sb}, if both variables positively influence each other, the observed association between the explanatory variable X and the outcome variable Y reflects the cumulative effects in both directions. Consequently, this simultaneity bias can lead to an overestimation of the causal effects of X on Y. 

\begin{figure}[h!]
    \centering
    \caption{Simultaneity bias}
    \label{fig:sb}
    \begin{tikzpicture}[thick]
        \node (X) {X};
        \node (Y) [right = of X] {Y};
          
        \draw[->] (X) -- (Y);
        \draw[->] (Y) -- (X);
    \end{tikzpicture}
\end{figure}

In the context of digital literacy and health, simultaneity bias presents in two potential scenarios. Firstly, poor health conditions may lead to lower level of digital literacy. Regrading physical health, some diseases can impede an individual's capacity to learn and understand digital skills and knowledge. For instance, it is well-documented that cancer and its treatments often lead to cognitive impairments, affecting memory, processing speed, attention, language, and executive functioning (Giovagnoli, 2012; Von Ah et al., 2013). Such cognitive declines can significantly hamper the ability of cancer patients, particularly older adults, to learn digital skills in a world where the internet still represents a relatively new challenge that demands considerable cognitive engagement. Regarding mental health, research shows that individuals with depression may experience symptoms such as anhedonia, which not only diminishes the desire to engage in new learning experiences but is also associated with dopaminergic dysfunction in the basal ganglia, further impairing the learning process (Herzallah et al., 2013). Moreover, conditions like schizophrenia, which can involve hallucinations, might prevent affected individuals from effectively using ICTs and even lead to challenges in recalling how to use such technologies (Greer et al., 2019).

Secondly, poorer health conditions may paradoxically lead to higher level of digital literacy. This arises because health limitations may restrict individuals' ability to perform everyday tasks in conventional ways (e.g., through face-to-face interactions), compelling them to rely more heavily on the Internet to meet their needs (Domingo, 2012). This increased dependency on digital solutions can motivate them to enhance their digital skills. For example, individuals with mobility issues might find it challenging to do grocery shopping physically and thus might need to develop the digital skills required to order groceries online (Dobransky \& Hargittai, 2006). This necessity drives them to become more proficient in digital tools and platforms, thereby inadvertently improving their digital literacy (Drentea et al., 2008). 

Most existing studies, as discussed in Section 2.1.2, adopt a ``selection on observables" approach, controlling for potential confounders to address omitted variable bias. However, this approach does not account for simultaneity bias, which can significantly affect the estimated relationships. Thus, findings from these studies still present a biased estimate of the causal effect of digital literacy on health, underscoring the need for more sophisticated analytical techniques to fully unravel these complex interactions.

\subsection{Range of health outcomes}
As noted in Section 2.1.1, the majority of existing research has primarily concentrated on exploring the impact of digital literacy on self-reported health and mental health outcomes. However, the aspect of physical health, which constitutes a crucial component of an individual's overall health, has received very limited attention. To date, the literature has only covered two physical health outcomes: diabetes (Guo et al., 2021) and lung disease (COPD) (Stellefson et al., 2019). This narrow scope overlooks a wide range of common physical health issues, including cardiovascular diseases such as hypertension, as well as non-cardiovascular chronic conditions like arthritis and cancer. These prevalent health concerns, given their significant impact on public health, warrant a more thorough examination to understand how digital literacy might influence their management and outcomes. This gap in the research highlights a critical need for broader investigative efforts to assess the role of digital literacy across a more diverse spectrum of physical health conditions.

\section{Limitations of existing studies - RQ2}
The findings from the existing literature on whether increased digitalisation will exacerbate the health disparities between older adults with low and high levels of digital literacy should be evaluated considering the following limitations. 

\subsection{Lack of quantitative evidence}
As highlighted in Section 2.2, the body of existing research on this question is notably scant, with only one study by Alkureishi et al. (2021), providing only qualitative evidence through interviews. Sociology, particularly in public health contexts, is fundamentally a population science (Goldthorpe, 2016, p. 31). Goldthorpe emphasises that the probabilistic nature of public health phenomena necessitates quantitative methods to adequately address the variability and heterogeneity within populations and to elucidate patterns and explanations in diverse contexts. The qualitative approach often faces limitations due to small sample sizes, for instance, the study by Alkureishi et al. (2021) only included 54 participants. This small sample size may not may not adequately represent broader population dynamics or reveal genuine patterns (Trafimow, 2014). Consequently, it is imperative to supplement these findings with quantitative research that can leverage larger, more diverse samples to produce more robust and generalisable evidence.

Furthermore, investigating the effects of increased digitalisation on health disparities requires a longitudinal approach, as the term "increasing" implies the need to compare populations at two or more distinct time points. While qualitative longitudinal studies exist and provide valuable insights, they often struggle with consistently comparing data across multiple time points due to the nature of qualitative analysis (Hermanowicz, 2013). In contrast, longitudinal quantitative studies offer a more feasible approach for such analysis. These studies allow for the consistent and quantifiable measurement of changes over time, providing stronger, more reliable estimates that can be effectively compared across different time points (Trafimow, 2014). 

\subsection{Causal inference}
The nature of the second research question - compare health outcomes between two groups at two time points - intuitively warrants a Difference-in-Difference (DiD) design, that is taking firstly the difference between time points and secondly taking the difference between groups (Angrist \& Pischke, 2009, p. 229). Consider a scenario where we conduct a longitudinal survey among older adults: at Wave 1, we assess their health conditions and categorise them into two groups based on their digital literacy levels. Assume that the level of digitalisation increases between Wave 1 and Wave 2. We then revisit and assess their health conditions a few years later (Wave 2). However, simply applying a Difference-in-Difference analysis may not yield an accurate estimate of the effect of increased digitalisation on the health disparities between groups with low and high digital literacy.

Firstly, there may be presence of group contamination due to the fact that digital literacy is a not a static attribute and can evolve over time (Chetty et al., 2018). As respondents' digital literacy can improve or decline between assessment waves, individuals initially categorised as having low digital literacy at Wave 1 might achieve higher levels by Wave 2. This fluidity in digital literacy levels can lead to contamination of group assignments, which in turn complicates the analysis. Consequently, the differences observed between Wave 1 and Wave 2 might not accurately represent the genuine effects of increased digitalisation on older adults' health outcomes.

Secondly, the presence of inherent heterogeneity between the two groups can violate the parallel trends assumption required in DiD analysis. The parallel trends assumption posits that, in the absence of the treatment, the treatment and control groups would have followed parallel paths over time (Angrist \& Pischke, 2009, p. 230). In this context, it implies that, without an increase in digitalisation, older adults with low and high digital literacy would have experienced similar health trends. However, this assumption may not hold in a straightforward DiD setup in this case because individuals with varying levels of digital literacy often differ significantly in socio-economic characteristics, which can influence their health trends independently of any changes in digitalisation. For example, digital literacy is often correlated with educational attainment, where those with lower digital literacy likely have lower educational levels. Meanwhile, lower educational levels are also linked to poorer health outcomes (Hall et al., 2015). Thus, without any changes in digitalisation, the group with lower digital literacy could still experience a more rapid health decline compared to their higher digital literacy counterparts. This discrepancy indicates that a simple DiD estimate may not accurately reflect the true impact of increased digitalisation on health disparities. 
